%
% File acl2019.tex
%
%% Based on the style files for ACL 2018, NAACL 2018/19, which were
%% Based on the style files for ACL-2015, with some improvements
%%  taken from the NAACL-2016 style
%% Based on the style files for ACL-2014, which were, in turn,
%% based on ACL-2013, ACL-2012, ACL-2011, ACL-2010, ACL-IJCNLP-2009,
%% EACL-2009, IJCNLP-2008...
%% Based on the style files for EACL 2006 by 
%%e.agirre@ehu.es or Sergi.Balari@uab.es
%% and that of ACL 08 by Joakim Nivre and Noah Smith

\documentclass[11pt,a4paper]{article}
\usepackage[hyperref]{acl2019}
\usepackage{times}
\usepackage{latexsym}
\usepackage{booktabs}
\usepackage{graphicx}
\usepackage{url}
\usepackage{xcolor}
%\aclfinalcopy % Uncomment this line for the final submission
%\def\aclpaperid{***} %  Enter the acl Paper ID here

%\setlength\titlebox{5cm}
% You can expand the titlebox if you need extra space
% to show all the authors. Please do not make the titlebox
% smaller than 5cm (the original size); we will check this
% in the camera-ready version and ask you to change it back.

\newcommand\BibTeX{B\textsc{ib}\TeX}

\title{The Language of Legal and Illegal Activity on DarkNet}

%\author{First Author \\
%  Affiliation / Address line 1 \\
%  Affiliation / Address line 2 \\
%  Affiliation / Address line 3 \\
%  \texttt{email@domain} \\\And
%  Second Author \\
%  Affiliation / Address line 1 \\
%  Affiliation / Address line 2 \\
%  Affiliation / Address line 3 \\
%  \texttt{email@domain} \\}

\date{}

\begin{document}
\maketitle
\begin{abstract}
  
\end{abstract}



%%%%%%%%%%%%%%%%%%%%%%%%%%%%%%%%%%%%%%%%%%%%%
\section{Introduction}

Omri

-- darknet has much illegal activity (https://www.profwoodward.org/2016/02/how-much-of-tor-is-used-for-illegal.html). people use it because it's easier to be anonymous, hard to track etc.

-- scalably monitoring activity in darknet uses NLP tools, but little is known about what characteristics the text in Tor has, and how well do off-the-shelf NLP tools do on this domain.

-- X et al. (2017) published a corpus and looked into text classification on texts from Tor, and Y et al. (2019) looked into how influential criminal sites are. 

-- We investigate how legal and illegal activity taken from DarkNet is different, comparing to a clearnet website with similar content as a control condition.

-- We also explore methods for classifying texts from DarkNet into legal and illegal. This is important both for understanding whether these two types are different in terms of
their text and in what ways, and as a practical tool.

-- A bit on the experiments and results.

%%%%%%%%%%%%%%%%%%%%%%%%%%%%%%%%%%%%%%%%%%%%%
\section{Related Work}

Elior

\paragraph{Detecting illegal activities in the Web}

The detection of illegal activities in the Web is sometimes derived from a more general topic classification. For example, \citet{Biryukov14} used the software Mallet \citep{McCallum02} and the web service uClassify \citep{Kagstrom13} for a classification of the content of Tor hidden services into 18 categories, which allows the distinction between illegal or contreversial content on one hand and human rights or freedom of speech content on the other hand. \citet{GraczykKinningham15} combined unsupervised feature selection and an SVM classfier for the classification of drug sales in an anonymous marketplace. However, although the detection of illegal activities can be easily deduced in some cases, the legal status of a given product can change \citep{GraczykKinningham15} and a given topic could cover both legal and illegal content. For example, in the recent work of \citet{Avarikioti18} on Tor content clasification, in most of the categories both legal and illegal content appear.

Some works have directly addressed a specific type of illegality and a particular communication context. \citet{MorrisHirst12} have used an SVM classification to identify sexual predators in chatting message systems. The model includes both lexical features, including emoticons and behavioral features that correspond to conversational patterns. Another example is the detection of pedophile activity in peer-to-peeer networks \citep{Latapy13} where a predefined list of keywords was used to detect child-pornography queries.





%%%%%%%%%%%%%%%%%%%%%%%%%%%%%%%%%%%%%%%%%%%%%
\section{Datasets}\label{sec:data}

DUTA \citep{AlNabki17}

\paragraph{Cleaning.} Elior + Daniel

As preprocessing for all experiments, we apply some cleaning to the text
of web pages in our corpus.
HTML markup is already removed in the original dataset,
but many nonlinguistic content remains, such as
buttons, encryption keys, metadata and URLs.
We remove such text from the web pages, and also join paragraphs to single lines
(as newlines are sometimes present in the original dataset for display purposes
only).
We then remove any duplicate paragraphs, where paragraphs are considered
identical if they share all but numbers
(to avoid an over-representation of some remaining surrounding text from the
websites, e.g. ``Showing all 9 results Search for
...'').\footnote{Our preprocessing code is included in the supplementary
material, and will be publicly available upon publication.}

%%%%%%%%%%%%%%%%%%%%%%%%%%%%%%%%%%%%%%%%%%%%%
\section{Domain Analysis}

\subsection{Distances between Domains}
\begin{table}[]
	\resizebox{\columnwidth}{!}{%
		\begin{tabular}{@{}lllll@{}}
			& all onion  & ebay       & illegal    & legal      \\
			all onion &            & 0.60, 1.31 & 0.33, 0.62 & 0.35, 0.65 \\
			ebay      & 0.60, 1.31 &            & 0.59, 1.28 & 0.66, 1.46 \\
			illegal   & 0.33, 0.62 & 0.59, 1.28 &            & 0.61, 1.28 \\
			legal     & 0.35, 0.65 & 0.66, 1.46 & 0.61, 1.28 &           
		\end{tabular}%
	}
	\caption{Jensen-Shannon divergence and Variational distance between word distribution in all onion drug sites, legal and illegal onion drug sites, and Ebay sites. \label{ta:domain}}
\end{table}
\begin{table*}[]
	\resizebox{\textwidth}{!}{%
		\begin{tabular}{@{}lllllllllllll@{}}
			& all onion  & all onion\_half1 & all onion\_half2 & ebay       & ebay\_half1 & ebay\_half2 & illegal    & illegal\_half1 & illegal\_half2 & legal      & legal\_half1 & legal\_half2 \\
			all onion        &            & 0.23, 0.34       & 0.25, 0.38       & 0.60, 1.31 & 0.61, 1.35  & 0.61, 1.35  & 0.33, 0.62 & 0.39, 0.77     & 0.41, 0.81     & 0.35, 0.65 & 0.41, 0.82   & 0.42, 0.82   \\
			all onion\_half1 & 0.23, 0.34 &                  & 0.43, 0.73       & 0.60, 1.32 & 0.62, 1.35  & 0.62, 1.35  & 0.37, 0.63 & 0.33, 0.63     & 0.50, 0.96     & 0.40, 0.70 & 0.36, 0.70   & 0.52, 1.00   \\
			all onion\_half2 & 0.25, 0.38 & 0.43, 0.73       &                  & 0.61, 1.34 & 0.62, 1.36  & 0.62, 1.37  & 0.39, 0.69 & 0.50, 0.96     & 0.35, 0.67     & 0.39, 0.67 & 0.51, 0.99   & 0.35, 0.66   \\
			ebay             & 0.60, 1.31 & 0.60, 1.32       & 0.61, 1.34       &            & 0.23, 0.36  & 0.25, 0.39  & 0.59, 1.28 & 0.60, 1.30     & 0.60, 1.32     & 0.66, 1.46 & 0.67, 1.48   & 0.67, 1.49   \\
			ebay\_half1      & 0.61, 1.35 & 0.62, 1.35       & 0.62, 1.36       & 0.23, 0.36 &             & 0.43, 0.74  & 0.60, 1.32 & 0.61, 1.33     & 0.61, 1.34     & 0.67, 1.48 & 0.67, 1.48   & 0.68, 1.50   \\
			ebay\_half2      & 0.61, 1.35 & 0.62, 1.35       & 0.62, 1.37       & 0.25, 0.39 & 0.43, 0.74  &             & 0.60, 1.30 & 0.61, 1.32     & 0.61, 1.32     & 0.67, 1.49 & 0.68, 1.50   & 0.68, 1.50   \\
			illegal          & 0.33, 0.62 & 0.37, 0.63       & 0.39, 0.69       & 0.59, 1.28 & 0.60, 1.32  & 0.60, 1.30  &            & 0.23, 0.35     & 0.27, 0.42     & 0.61, 1.28 & 0.62, 1.31   & 0.62, 1.31   \\
			illegal\_half1   & 0.39, 0.77 & 0.33, 0.63       & 0.50, 0.96       & 0.60, 1.30 & 0.61, 1.33  & 0.61, 1.32  & 0.23, 0.35 &                & 0.45, 0.77     & 0.62, 1.31 & 0.63, 1.33   & 0.62, 1.32   \\
			illegal\_half2   & 0.41, 0.81 & 0.50, 0.96       & 0.35, 0.67       & 0.60, 1.32 & 0.61, 1.34  & 0.61, 1.32  & 0.27, 0.42 & 0.45, 0.77     &                & 0.62, 1.31 & 0.63, 1.33   & 0.63, 1.33   \\
			legal            & 0.35, 0.65 & 0.40, 0.70       & 0.39, 0.67       & 0.66, 1.46 & 0.67, 1.48  & 0.67, 1.49  & 0.61, 1.28 & 0.62, 1.31     & 0.62, 1.31     &            & 0.26, 0.40   & 0.26, 0.42   \\
			legal\_half1     & 0.41, 0.82 & 0.36, 0.70       & 0.51, 0.99       & 0.67, 1.48 & 0.67, 1.48  & 0.68, 1.50  & 0.62, 1.31 & 0.63, 1.33     & 0.63, 1.33     & 0.26, 0.40 &              & 0.47, 0.82   \\
			legal\_half2     & 0.42, 0.82 & 0.52, 1.00       & 0.35, 0.66       & 0.67, 1.49 & 0.68, 1.50  & 0.68, 1.50  & 0.62, 1.31 & 0.62, 1.32     & 0.63, 1.33     & 0.26, 0.42 & 0.47, 0.82   &             
		\end{tabular}%
	}
	\caption{Jensen-Shannon divergence and Variational distance between word distribution in all onion drug sites, legal and illegal onion drug sites, and Ebay sites, each domain was also split in half for in domain comparison. \label{ta:domain_halves}}
	
\end{table*}
At this point, we strive to know which of the domains we deal with is a different domain in terms of word use, whether all onion language should be considered the same, and is there a potential in learning differences between onion sites. For those means, we created a histogram of the frequencies of words in the Ebay corpus and the legal and illegal drugs onion corpora, together with the combination of the two latter, representing all drug sites in onion. Additionally, each corpus was randomly split in two halves allowing for in-domain comparison. Following \cite{Plank2011EffectiveMO} we chose Jensen-Shannon divergence and Variational distance (also known as L1 or Manhattan) as the comparison measures between the word frequency histograms.

We see in Table \ref{ta:domain_halves} that ebay, legal and illegal are self-distant 0.4-0.45 but the distance between one another is 0.6-0.65. This means the three form an equilateral. Which in turn means that we can think of the three as three different domains. The results suggest there is no reason to think about all onion as one domain, and thus studying using onion data what characterizes the illegal domain from the legal domain is sensible. We also see that Ebay is as far from the legal domain as it is from the illegal or for that case all onion domain, this would suggest there are also unique characteristics for each domain which will make learning illegal properties rather than simply domain differences more challenging. For this reason if one wishes to understand differences that stem from legality issues, it might prove more beneficial to differentiate illegal and legal inside onion domain, rather than world wide web legal domains such as ebay.

\subsection{NER and Wikification}

In order to analyze the named entities in each domain and the differences
between them we used a ``wikification'' technique, searching for
said entities in public datasets such as Wikipedia. 

Using spaCy's\footnote{\url{https://spacy.io}}
named entity recognition, we first found all the named
entities in each site in the datasets. After finding the named entites
we searched for relevant Wikipedia entries for each named entity using
the DBpedia Ontology API. For each domain we counted the total number
of named entites and what percentage of them had corresponding Wikipedia
articles.

\begin{table}

\caption{Percentage of Wikifiable Named Entites per Domain}

\begin{centering}
\begin{tabular}{|c||c|}
\hline 
 & Percentage\tabularnewline
\hline 
\hline 
Legal Onion & 50.8 $\pm2.31$\tabularnewline
\hline 
\hline 
Illegal Onion & 32.5 $\pm1.35$\tabularnewline
\hline 
\hline 
eBay & 38.6 $\pm2$\tabularnewline
\hline 
\end{tabular}
\par\end{centering}
\end{table}
According to our results (Table 1) the wikification percentages of
eBay sites and illegal Onion sites are comparable and relatively low.
However, sites selling legal drugs on Onion have a much higher wikification
percentage.

Presumably the named entities in Onion sites selling legal drugs are
more easily found in public databases such as Wikipedia because they
are mainly well known names for legal pharmaceuticals. However, in
both illegal Onion and eBay sites, the list of named entities include
many nicknames for illicit drugs and paraphernalia. These nicknames
are usually not well known by the general public and are therefore
less likely to be found on Wikipedia or other public databases.

In addition to the differences in wikification percentages between
the domains, we found that spaCy had trouble correctly identifying
named entities in both Onion and eBay sites. There were a fair number
of false positives (words and phrases that were found by spaCy but
were not actually named entities), especially in illegal Onion sites.
We believe the informal language used in our datasets makes it harder
for spaCy to function optimally in this capacity.

These findings lead us to believe that the popular tools used today
for named entity recognition and analysis are not ideal for processing
informal language on the internet, especially language dealing in
illicit activities. 

%%%%%%%%%%%%%%%%%%%%%%%%%%%%%%%%%%%%%%%%%%%%%
\section{Classification Experiments}

Daniel

\paragraph{Experimental setup.}

After cleaning the dataset, joining lines to paragraphs and removing duplicates
(see \S\ref{sec:data}), we split each subset into training, validation and test.
We choose to have 456 training paragraphs, 57 validation paragraphs and
58 test paragraphs (approximately a 80\%/10\%/10\% split) for each category,
thus randomly downsampling larger categories for an even division of labels.

\paragraph{Model.}

To classify paragraphs into categories, we experiment with five classifiers:

\begin{itemize}
  \item Naive Bayes bag-of-words: we use \texttt{BernoulliNB} from
  \texttt{scikit-learn}\footnote{\url{https://scikit-learn.org}}
  with $\alpha=1$.
  \item SVM: we use \texttt{SVC}, also from \texttt{scikit-learn},
  with $\gamma=$``scale'' and tolerance=$10^{-5}$.
  \item BoE (bag-of-embeddings): we represent each word with its 100-dimensional
  GloVe vector \cite{pennington2014glove}, average the embeddings for all words in the paragraph
  to a single vector, and apply a 100-dimensional fully-connected layer with
  ReLU non-linearity and dropout $p=0.2$.
  The word vectors are not updated during training.
  \item seq2vec: same as BoE, but instead of averaging word vectors,
  we apply a BiLSTM to the word vectors, and take the concatenated
  final hidden vectors from the forward and backward part as the input to the
  fully-connected layer.
  \item ELMo+attention: we replace the word representations with contextualized
  pre-trained represetations from ELMo \cite{Peters:2018}. We then apply a self-attentive
  classification network \cite{mccann2017learned} over the contextualized representations.
\end{itemize}

We use the AllenNLP Python library\footnote{\url{https://allennlp.org}}
\cite{Gardner2017AllenNLP} for implementing the neural network classifiers.

\paragraph{Data manipulation.}

To check the sensitivity of classifiers to variations in content words,
function words and syntax, we experiment with five manipulations to the input
data (in training, validation and testing).
For this purpose, we consider content words as words whose universal part-of-speech
according to spaCy is one of \{ADJ, ADV, NOUN, PROPN, VERB, X, NUM\},
and function words as all other words.
The tested manipulations are:

\begin{itemize}
  \item Dropping all content words.
  \item Dropping all function words.
  \item Replacing all content words with their universal part-of-speech.
  \item Replacing all function words with their universal part-of-speech.
  \item Replacing all words with their universal part-of-speech.
\end{itemize}

\subsection{Results}

\paragraph{Legal vs. Illegal}

\paragraph{Legal Onion vs. Ebay}


%%%%%%%%%%%%%%%%%%%%%%%%%%%%%%%%%%%%%%%%%%%%%
\section{Discussion}

The legal and illegal are pretty distant, which is evident in a few ways: word distribution is different, NER and Wikification work less well
for illegal. This has practical implications: we need to adapt our tools to deal with illegal Tor data. 

Looking at specific sentences, we see that it's hard distinguishing them based on the identity of the words, which means that
looking at the wordforms is a very poor solution for tackling this. However, using modern text classification, they can be distinguished
in a 72\%. 

Looking at how different types of language influence results: given that replacing all words with their POS tags gives
the same performance, this tells us their syntax is different as well.

Methological: Legal and illegal in Onion are distinct enough to be considered different domains (as distant as Legal and Ebay). 
Therefore, Tor could be a good testbed for working on legal and illegal classification.


%%%%%%%%%%%%%%%%%%%%%%%%%%%%%%%%%%%%%%%%%%%%%
\section{Conclusion}
 












\bibliography{acl2019}
\bibliographystyle{acl_natbib}




\end{document}
