%
% File acl2019.tex
%
%% Based on the style files for ACL 2018, NAACL 2018/19, which were
%% Based on the style files for ACL-2015, with some improvements
%%  taken from the NAACL-2016 style
%% Based on the style files for ACL-2014, which were, in turn,
%% based on ACL-2013, ACL-2012, ACL-2011, ACL-2010, ACL-IJCNLP-2009,
%% EACL-2009, IJCNLP-2008...
%% Based on the style files for EACL 2006 by 
%%e.agirre@ehu.es or Sergi.Balari@uab.es
%% and that of ACL 08 by Joakim Nivre and Noah Smith

\documentclass[11pt,a4paper]{article}
\usepackage[hyperref]{acl2019}
\usepackage{times}
\usepackage{latexsym}

\usepackage{url}

%\aclfinalcopy % Uncomment this line for the final submission
%\def\aclpaperid{***} %  Enter the acl Paper ID here

%\setlength\titlebox{5cm}
% You can expand the titlebox if you need extra space
% to show all the authors. Please do not make the titlebox
% smaller than 5cm (the original size); we will check this
% in the camera-ready version and ask you to change it back.

\newcommand\BibTeX{B\textsc{ib}\TeX}

\title{The Language of Legal and Illegal Activity on Tor}

%\author{First Author \\
%  Affiliation / Address line 1 \\
%  Affiliation / Address line 2 \\
%  Affiliation / Address line 3 \\
%  \texttt{email@domain} \\\And
%  Second Author \\
%  Affiliation / Address line 1 \\
%  Affiliation / Address line 2 \\
%  Affiliation / Address line 3 \\
%  \texttt{email@domain} \\}

\date{}

\begin{document}
\maketitle
\begin{abstract}
  
\end{abstract}



%%%%%%%%%%%%%%%%%%%%%%%%%%%%%%%%%%%%%%%%%%%%%
\section{Introduction}



%%%%%%%%%%%%%%%%%%%%%%%%%%%%%%%%%%%%%%%%%%%%%
\section{Related Work}

\paragraph{Detecting illegal activities in the Web}

The detection of illegal activities in the Web is sometimes derived from a more general topic classification. For example, \citet{Biryukov14} used the software Mallet \citep{McCallum02} and the web service uClassify \citep{Kagstrom13} for a classification of the content of Tor hidden services into 18 categories, which allows the distinction between illegal or contreversial content on one hand and human rights or freedom of speech content on the other hand. \citet{GraczykKinningham15} combined unsupervised feature selection and an SVM classfier for the classification of drug sales in an anonymous marketplace. However, although the detection of illegal activities can be easily deduced in some cases, the legal status of a given product can change \citep{GraczykKinningham15} and a given topic could cover both legal and illegal content. For example, in the recent work of \citet{Avarikioti18} on Tor content clasification, in most of the categories both legal and illegal content appear.

Some works have directly addressed a specific type of illegality. \citet{MorrisHirst12} have used an SVM classification to identify sexual predators in chatting message systems. The model includes both lexical and behavioral features.





%%%%%%%%%%%%%%%%%%%%%%%%%%%%%%%%%%%%%%%%%%%%%
\section{Datasets}

DUTA \citep{AlNabki17}

\paragraph{Cleaning.}

%%%%%%%%%%%%%%%%%%%%%%%%%%%%%%%%%%%%%%%%%%%%%
\section{Domain Analysis}

\subsection{Distances between Domains}

Leshem



\subsection{NER and Wikification}

Dan: experimental setup, results and 

%%%%%%%%%%%%%%%%%%%%%%%%%%%%%%%%%%%%%%%%%%%%%
\section{Classification Experiments}

Daniel

\subsection{Results}

\paragraph{Legal vs. Illegal}

\paragraph{Legal Onion vs. Ebay}


%%%%%%%%%%%%%%%%%%%%%%%%%%%%%%%%%%%%%%%%%%%%%
\section{Discussion}

The legal and illegal are pretty distant, which is evident in a few ways: word distribution is different, NER and Wikification work less well
for illegal. This has practical implications: we need to adapt our tools to deal with illegal Tor data. 

Looking at specific sentences, we see that it's hard distinguishing them based on the identity of the words, which means that
looking at the wordforms is a very poor solution for tackling this. However, using modern text classification, they can be distinguished
in a 72\%. 

Looking at how different types of language influence results: given that replacing all words with their POS tags gives
the same performance, this tells us their syntax is different as well.

Methological: Legal and illegal in Onion are distinct enough to be considered different domains (as distant as Legal and Ebay). 
Therefore, Tor could be a good testbed for working on legal and illegal classification.


%%%%%%%%%%%%%%%%%%%%%%%%%%%%%%%%%%%%%%%%%%%%%
\section{Conclusion}
 












\bibliography{acl2019}
\bibliographystyle{acl_natbib}




\end{document}
